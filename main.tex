% !TEX TS-program = pdflatex
% !TEX encoding = UTF-8 Unicode

% This is a simple template for a LaTeX document using the "article" class.
% See "book", "report", "letter" for other types of document.

\documentclass{article} 
\usepackage[14pt]{extsizes} %задаем размер шрифта

\usepackage[utf8]{inputenc} % set input encoding (not needed with XeLaTeX)
\usepackage[T2A]{fontenc}
\usepackage[pages=some]{background}
\usepackage{graphicx} % support the \includegraphics command and options
\usepackage[most]{tcolorbox} % для управления цветом

\definecolor{sw}{RGB}{229,177,58} %Цвет текста звездных войн эпизода 1
\definecolor{sky_black}{RGB}{17,16,10} %Цвет ночного неба
% Настройка для рамки вокруг текста
\definecolor{block-gray}{gray}{0.90} % уровень прозрачности (1 - максимум)
\newtcolorbox{myquote}{colframe=sw, colback=black,grow to right by=-10mm,grow to left by=-10mm,
boxrule=2pt,boxsep=0pt} % настройки области с изменённым фоном

\backgroundsetup{
scale=1,
color=black,
opacity=1,
angle=0,
contents={%
  \includegraphics[width=\paperwidth,height=\paperheight]{imgs/stars_img.jpg}
  }%
}

\title{Забытый артефакт}
\author{SW Duncan's community}
%\date{} % Activate to display a give date or no date (if empty),
         % otherwise the current date is printed 

\color{sw}
\begin{document}
\pagecolor{sky_black}
\BgThispage
\maketitle %титульная страница
\newpage
\tableofcontents %оглавление
\newpage
\section{В поисках храма}

\begin{myquote}
%\pagecolor{black}
\color{sw}
Планета была похожа на результат столкновения двух капель краски: зеленой и синей. 
В результате на планете образовалось две цветный кляксы неровно притертый друг к другу,
 и не было видно ни одно цвета, кроме синего и зеленого на этой планете от экватора до полюса.
\end{myquote}
\subsection{Воздушный вояж}
\begin{myquote}
\color{sw}
Не заходя на посадку, вы решил искать храм по воздуху.
Спустив свой корабь к макушкам исполинских деревев, вы издали заметили несколько золмов из крон.
Летая от одного холма к другому вы находили древние огромные постройки, в текущий момент разрушенные почти до основания.
Однако на последней попытке вам повезло, под деревьями прятлась единственная целая постройка, ваша цель.
\end{myquote}


\begin{myquote}
\color{sw}
Не заходя на посадку, вы решил искать храм по воздуху.
Спустив свой корабь к макушкам исполинских деревев, вы издали заметили несколько холмов из крон.
Летая от одного холма к другому вы находили древние огромные постройки, в текущий момент разрушенные почти до основания.
Во время поиска, буквально за одну минуту, вы оказлись в тумане, по сути в не проглядной мгле.
Пошле пиел ветер, нет, ветром это можно было назвать, если спутать льва с домашней кошкой. 
На вас ашел ураган, что так свойственен этой непостоянной планете.
И ураган разрзился громом, отключив у всего корабля электроникиу.
Теперь вы были предоставлены только удаче.

Спустя пару часов в каруселе сильнейшего за вашу жизнь урагана вы разбились рядом со старым зднием.
Корабль больше не подходил для полетов.
Однако, когда небо просветлело, то вы поняли, что находитесь рядом с хрмом, который искали.
Хоть одна хорошая новость.
\end{myquote}
\subsection{С дипломатической миссией к каннибалам}
\begin{myquote}
\color{sw}
Медленно приближаясь, поверхность планеты превращалась в монолитные джунгли, создавая ощущение, что человек здесь никогда не был. Каждое дерево, казалось, рослосло более тысячи лет, в обхвате став болье, чем смогли бы вмести в свои руки 10 человек. С могучих толстых веток, больше подходящих дубовым стволам, спускались лианы тощиной в человеческую ногу. А где-то далеко кричало большое животное, спасающееся от хищников. Было понятно лишь одно - это опсное место. 
\end{myquote}

\begin{myquote}
\color{sw}
Сквозь сплошной покров начали пробиваться лучи солнца.
По непонятной причине вся територия впереди была безлесной.
Но это никак не значило, что там ничего растительного не было.
Из тонких веток, скрученных стеблей травы, огромных листьев были собраны несколько десятков домиков, средь которых резвилась смуглая ребятня, занимались рукоделием женщины, где-то кучковались мужчины. В центре же деревни стоял камень, похожий на вздернутый вверх палец, который был сверху обильно обагрен застарелой и свежей кровью.
\end{myquote}

\begin{myquote}
\color{sw}
Деревня выделила проводика до храма. Он спокойно вел учеников сквозь сплошное переплетение лиан, веток, деревьев. Ученики не смогли найти ни одного ориентира, который бы помог им вернуться. Однако майчок на корбле слёгкостью поможет вернуться.
\end{myquote}

\section{Храм лорда ситхов}

\begin{myquote}
\color{sw}
Мрачное, низкое строение, по верх которого растут низкорослые деревья, оплетая корнями обсидановые блоки храма. На поверхости стен выгравированны забатым, темным языком слова пророчеств и ритуалов. 
Меж строк изображены жервоприношений, создание конструктов из тел разумных и упражнения Силы. Выше по лестнице темнеет единственный проход.
\end{myquote}

\begin{myquote}
\color{sw}
Бродя по храму, вы наткнулись на комнату, полную злобных духов, возвращенных к жизни могучим ситхским лордом.
Пол комнаты покрыт застарелой кровью - явный признак темных ритуалов. Вероятно, описание самих ритуалов можно найти тут же, все стены комнаты исписаны странными символами, которые вы видели ранее.
Ваши шаги гулко раздаются по всему помещению, поэтому скрыться вряд ли получится.
Впереди видна массивная запертая дверь, за которой явно скрывается главная гробница. К сожалению, призраки не настроены вас туда пропустить.

\end{myquote}
Призракам не нанести урон бластерами или световыми мечами.
Бороться с ними придется на ментальном фронте, объединив усилия и создавая Волны тьмы, отправляющие призраков назад в Бездну.
Кроме того, вам может помочь Светлый дух, вызвав Стену света (если, конечно, вам не противно применение Светлой стороны в своем присутствии)
Если подпустить призрака слишком близко к себе, он сможет овладеть вашим разумом и свести с ума, так что будьте осторожны и следите за своими товарищами - любой из них может вонзить меч вам в спину.
\end{document}
